% add --shell-escape to pdflatex arguments.
% add following key to have keyboard shortcuts
%{
%    "key": "shift+b",
%    "command": "commandId",
%    "when": "editorTextFocus"
%},
%{
%"key": "shift+B",
%"command": "editor.action.insertSnippet",
%"when": "editorLangId == latex && editorTextFocus",
%"args": {
%    "snippet": "\\textbf{${TM_SELECTED_TEXT}$0}"
%}
%}

\documentclass[14pt]{extarticle}
\usepackage[left=2cm , right = 2cm, top=2cm]{geometry}
\usepackage{helvet}
\usepackage{parskip}
\usepackage{amsmath}
\usepackage{amssymb}
\usepackage{graphicx}
\usepackage[spanish]{babel}
\usepackage[dvipsnames]{xcolor}
\usepackage{tcolorbox} % above of the svg package
\usepackage{svg} 
\usepackage{hyperref}
\usepackage{minted}
\renewcommand{\sfdefault}{lmss}  % este activa la letra lmss
\renewcommand{\familydefault}{\sfdefault} % este activa la letra lmss
\sffamily % este activa la letra lmss

%\hyperlink{page.2}{Go to page 2}
%\newpage
%text on page 2
%\begin{figure}[htbp]
%  \centering
%  \includesvg{plot.svg}
%  \caption{svg image}
%\end{figure}

%\begin{minted}{csharp}
%    // single comment
%    \end{minted}

% f(n) = \begin{cases}
%    n/2  & n \text{ is even} \\
%    3n+1 & n \text{ is odd}
%  \end{cases}

%\begin{align}
%    \frac{d}{dx} \ln x &= \lim_{h\to 0} \frac{\ln(x+h) - \ln x}{h} \\
%    &= \ln e^{1/x} &&\text{How this follows is left as an exercise.}\\
%    &= \frac{1}{x} &&\text{Using the definition of ln as inverse function}
%   \end{align}

\begin{document}



@BritoAlv



\begin{tcolorbox}[colback=blue!5!white,colframe=blue!75!black, title = Preguntas]

\begin{enumerate}
    \item Dados $a,b,c \in  \mathbb{N}$ tales que $(a,b) = 1$ probar que existen infinitos $n \in \mathbb{N}$ tales que:
    $$(a+bn, c) = 1$$

    \item Probar que la función $\phi(n)$ es multiplicativa usando una vía alternativa a la demostración de rellenar una tabla con los números de $1, mn$.

    \item Marque $V,F$:
    \begin{itemize}
        \item Si $p \equiv 5 \mod{8} $ y $p | a^4+b^4$ entonces se cumple que $p | a$ y $p | b$.
        
        \item Él número $2^{2^n}-1$ es divisible por al menos $n$ números primos distintos.
    
        \item La suma de los elementos de un sistema residual reducido respecto a $n \in \mathbb{N}, n > 2$ es divisible por $n$.  
    \end{itemize}
\end{enumerate}   

\end{tcolorbox}

\newpage



\end{document}

