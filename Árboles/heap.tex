\documentclass[border=3mm]{standalone}
\usepackage[edges]{forest}
\usetikzlibrary{positioning}

\begin{document}
\tikzset{every label/.style={xshift=-4ex, text width=6ex, align=right, 
                             inner sep=1pt, font=\footnotesize, text=red}}
\begin{forest}
    for tree={              % style of tree nodes
    font=\footnotesize,
    draw, semithick, rounded corners,
          align = center,
      inner sep = 2mm,
                          % style of tree (edges, distances, direction)
           edge = {draw, semithick, -stealth},
  parent anchor = east,
   child anchor = west,
           grow = south,
  forked edge,            % for forked edge
          l sep = 12mm,   % level distance
       fork sep = 6mm,    % distance from parent to branching point
              }
        [1, label = 0
            [6, label = 1
            [7, label = 3]
            [10, label = 4]
            ]
            [7, label = 2
            [8, label = 5]
            [9, label = 6]]
        ]
        \node[below = 0.5cm of current bounding box.south]{\textbf{El array es $[1,6,7,7,10,8,9]$, la figura muestra como visualizarlo como un binary tree}};       
\end{forest}
\end{document}