\documentclass[margin=3mm]{standalone}
\usepackage[edges]{forest}
\usepackage{tikz}
\usetikzlibrary{positioning}
\usetikzlibrary{shadows,arrows.meta}
\tikzset{parent/.style={align=center,text width=3cm,fill=green!20,rounded corners=2pt},
    child/.style={align=center,text width=2.8cm,fill=green!50,rounded corners=6pt},
    grandchild/.style={fill=pink!50,text width=2.5cm}}
\begin{document}
\begin{forest}
    arrow label/.style={
        if level=1{
          edge label={node [midway, font=\scriptsize\sffamily, sloped, above] {#1}},
        }{
          edge label={node [midway, font=\scriptsize\sffamily, anchor=south west] {#1}},
        },
      },
    for tree={
    grow=south,
    draw, minimum size=3ex, inner sep=4pt,
    s sep=10mm, thick,
    drop shadow,
    l sep=1.4cm,
    s sep= 1cm,
    node options={draw,font=\sffamily},
    edge={semithick,-Latex}
    }
    [$n$ números a ordenar , name =L1
        [$<$ que pivote $O(m)$, arrow label=$m \leq n-1$ 
                [$<$ que pivote
                []
                [pivote]
                []]
                [pivote]
                [$>$ que pivote
                []
                [pivote]
                []
                ]
        ]
        [pivote, , before computing xy={s/.average={s}{siblings}}]
        [$>$ que pivote $O(t)$ , arrow label = $t \leq n-1$, name = L2 
                [$<$ que pivote
                []
                [pivote]
                []]
                [pivote]
                [$>$ que pivote, name = L3
                []
                [pivote]
                [,name = L4]
                ]
        ]
    ]
    \node (a) [right=of L1 -| L2.east] {$O(n)$};
    \node (b) [right=of L2.east]       {$m+t = n-1, O(m+t) = O(n)$};
    \node (c) [right=of L3.east]      {$\sum = n-2, O(n)$};
    \node (d) [right=of L4.east] {$\sum = n - CP , O(n)$};
    \draw[dashed,->] (L1) -- (a);
    \draw[dashed,->] (L2) -- (b);
    \draw[dashed,->] (L3) -- (c);
    \draw[dashed,->] (L4) -- (d);
    \node[below = 0.5cm of current bounding box.south]{\textbf{$CP$ representa la cantidad de pivotes que fueron determinados, notemos que si alguno de $m,t$ es $0$, sucede que se poda esa rama del árbol, de ahí que el peor caso tenga complejidad $O(n^2)$}};       
\end{forest} 
\end{document}