% add --shell-escape to pdflatex arguments.
% add following key to have keyboard shortcuts
%{
%    "key": "shift+b",
%    "command": "commandId",
%    "when": "editorTextFocus"
%},
%{
%"key": "shift+B",
%"command": "editor.action.insertSnippet",
%"when": "editorLangId == latex && editorTextFocus",
%"args": {
%    "snippet": "\\textbf{${TM_SELECTED_TEXT}$0}"
%}
%}

\documentclass[14pt]{extarticle}
\usepackage[left=2cm , right = 2cm, top=2cm]{geometry}
\usepackage{helvet}
\usepackage{parskip}
\usepackage{amsmath}
\usepackage{amssymb}
\usepackage{graphicx}
\usepackage[spanish]{babel}
\usepackage[dvipsnames]{xcolor}
\usepackage{tcolorbox} % above of the svg package
\usepackage{svg} 
\usepackage{hyperref}
\usepackage{minted}
\renewcommand{\sfdefault}{lmss}  % este activa la letra lmss
\renewcommand{\familydefault}{\sfdefault} % este activa la letra lmss
\sffamily % este activa la letra lmss
%\hyperlink{page.2}{Go to page 2}
%\newpage
%text on page 2
%\begin{figure}[htbp]
%  \centering
%  \includesvg{plot.svg}
%  \caption{svg image}
%\end{figure}

%\begin{minted}{csharp}
%    // single comment
%    \end{minted}

%\begin{align}
%    \frac{d}{dx} \ln x &= \lim_{h\to 0} \frac{\ln(x+h) - \ln x}{h} \\
%    &= \ln e^{1/x} &&\text{How this follows is left as an exercise.}\\
%    &= \frac{1}{x} &&\text{Using the definition of ln as inverse function}
%   \end{align}


\begin{document}



@BritoAlv



\begin{tcolorbox}[colback=blue!5!white,colframe=blue!75!black, title = IMO2021P1]

    Let $n >= 100$  be an integer. The numbers $n, n+1, . . . , 2n$ are written on $n+1$ cards,
    one number per card. The cards are divided into two piles. Prove that one of the
    piles contains two cards such that the sum of their numbers is a perfect square.

\end{tcolorbox}

Viendo el problema desde el punto de vista de un grafo donde dos números están conectados si suman cuadrado perfecto, es suficiente probar que en $[n,2n]$ hay un ciclo de longitud impar.

La condición del problema puede ser aflojada demostrando que existe un ciclo de longitud 3 siempre. Por lo que hace falta hallar tres enteros que sumen cuadrado perfecto dos a dos:

$$a+b = p^2, b+c = q^2, c+a = r^2$$

La condición del problema nuevamente puede ser aflojada demostrando que para $n \geq 100$ se puede escoger como $(p,q,r) = (x-1, x , x+1 )$. Demostraré lo último.

Queda que
$$
    a = \frac{(x-1)^2 + (x)^2 - (x+1)^2}{2} = \frac{x^2-4x}{2}$$
$$
    b  = \frac{(x-1)^2 + (x+1)^2 - (x)^2}{2} = \frac{x^2+2}{2}$$
$$
    c =\frac{(x+1)^2 + (x)^2 - (x-1)^2}{2} = \frac{x^2+4x}{2}
$$

Por lo que queda hallar los $x$ para los cuales:

$$\frac{x^2-4x}{2} \geq n $$
$$\frac{x^2+4x}{2} \leq 2n $$

Obtenemos que:
$$
    \sqrt{2n+4}+2 < x < \sqrt{4n+4}-2
$$

Para $x \geq 100$ hay al menos dos valores que satisfacen lo anterior, escogemos el par, para que $a,b,c$ sean enteros y listo.

\newpage

\begin{tcolorbox}[colback=blue!5!white,colframe=blue!75!black, title = ?]

    Let $f$ be a polynomial with integer coefficients, of degree $n > 1$. What's the maximal number of consecutive integers belonging to the sequence $f(1), f(2), ... ?$.


\end{tcolorbox}

How do we use the fact that the degree is not $1$, how to ensure that integer coefficients imply absolute difference between any two value greater than $1$, indeed $a-b \vert f(a) - f(b)$, so if we have for example 



\end{document}

